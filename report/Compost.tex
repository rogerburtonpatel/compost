\documentclass[11pt]{article}
\usepackage{amssymb}
\usepackage{amsmath}
\usepackage{amsthm}
%\usepackage{fullpage}
\usepackage{graphicx}
%\usepackage{caption}
\usepackage{hyperref}
\usepackage{listings}
\usepackage{changepage}
\usepackage{syntax}

\newcommand\op{\texttt{(}}
\newcommand\cl{\texttt{)}}

\renewcommand{\syntleft}{\normalfont\itshape}
\renewcommand{\syntright}{}

\setlength{\grammarparsep}{20pt plus 1pt minus 1pt} % increase separation between rules
\setlength{\grammarindent}{12em} % increase separation between LHS/RHS

\setlength{\parindent}{0cm}

\title{Pure Garbage: The Compost Final Report}

\author{}

\date{December 15, 2023}

\begin{document}

\maketitle

\begin{verbatim}
val name_email_map : (string * string) list =
[("Roger Burtonpatel", "roger.burtonpatel@tufts.edu");
 ("Randy Dang", "randy.dang@tufts.edu");
 ("Jasper Geer", "jasper.geer@tufts.edu");
 ("Jackson Warhover", "jackson.warhover@tufts.edu")]  
\end{verbatim}


% INTRODUCTION

\section{Introduction} {
    Explain the purpose and motivation of your language, its use cases, and any necessary background that I'll need to 
    understand the point of your language.
}

% LANGUAGE TUTORIAL

\section{Language Tutorial} {
    A short explanation telling a novice how to use your language (consider this an informal version of a Language 
    Reference Manual). 
    \\\\ 
    Explains how to use the compiler in its simplest form and run a compiled program.
    \\\\
    Incrementally introduces how the language works through informal, well-documented code examples.
}

% LRM

\section{Language Manual} {
    Include your language reference manual. Make sure it's been updated if you've made \textit{any} changes since the first 
    LRM deliverable was turned in. I \textbf{will} use this to try to write my own programs in your language!
}

% PROJECT PLAN

\section{Project Plan} {
    Identify process used for planning, specification, and development
    \\\\
    Show your project timeline
    \\\\
    Identify roles/responsibilities/contributions of each team member
    \\\\
    Describe the software development environment used (tools and languages)
    \\\\
    If possible, include a visualization of version control commits (but not a dump of a commit log)
}

% ARCHITECTURAL DESIGN

\section{Architectural Design} {
    Give block diagram showing the major components of your compiler and the interfaces between them
    \\\\
    Summarize how the language's "interesting" features were implemented
    \\\\
    State who implemented each component
}

% TEST PLAN

\section{Test Plan} {
    Explain how your group approached unit and integration testing, and what automation was used.
    \\\\
    Show two or three representative source language programs along with the target language program 
    generated for each (if you can provide syntax highlighting and nice formatting that's REALLY useful)
    \\\\
    State who did what
}

% LESSONS LEARNED

\section{Lessons Learned} {
    Each team member should explain their most important takeaways from working on this project
    \\\\
    Include any advice the team has for future teams
}

% APPENDIX 

\section{Appendix} {
    Attach a complete code listing of your translator with each module signed by its author(s)
    \\\\
    Do not include any automatically generated files, only the sources.
}

\end{document}
